\section{Conclusions}
The following look at the results achieved by the project and future opportunities.

\subsection{Appraisal}
The final application created is believed by the student to have successfully hit the initial goals set forth (See \ref{ressummary}). The application created is a viable market entry that simplifies the creation of scatterplots and offers a wide range of controls. The only major shortcoming according to the initial goals would be the lower focus on Big Data capability and support. The Student believes that this was nevertheless a correct decision, focusing on the basic visualization and accessibility has lead to the creation of a usable application that is in a good position to be further developed to fully support Big Data- Something that would have been much more complex to do from the start. Even at the current state of the application there is still some support for 2 properties \cite[]{6612229} of Big Data:
\begin{itemize}
    \item Size - The application is capable of rendering large datasets and the slice view design allows a smaller subset of data to be rendered at a time to avoid resource strain.
    \item Complexity -  Up to 4 dimensions can be viewed and relationships compared, the student believes that this is helpful to addressing some complexity of data, but further design and research opportunities exist to address other complexity aspects. \cite[]{6612229}
\end{itemize}

Another minor shortcoming would be that no other graph type was implemented- It would have been nice to allow the user to view the data through different graph types but it wouldn't have been possible to provide as many controls as for a single-focused charting application. Though with the abstractions in place created for scatter plots, future addition of new graphs would be much simpler and faster.

In terms of preparation, Prototyping was key for this project and allowed the student to practice using unfamiliar tools and debug the majority of setup issues before development started- which allowed the student to start development focusing on user stories instead of debugging issues. The creation of prototype 2 also greatly sped up development by getting basic rendering functionality done further allowing greater focus on user stories. (See \ref{prototype} for more details)

PXP also set a good base agile methodology (See \ref{devmet} for more details) that the student could restructure to fit the project. Those changes all helped the student to work more efficiently and greatly helped with tracking project progress. Though the inability to apply automated testing to rendering was a definite limitation. For a future project, it would be great to allocate some time for a prototype to figure out a Selenium based or otherwise technique to create general purpose tests that would ideally need as little change as possible on running. (See \ref{testframe} for more details)

As for the process of creating the application, The student believes that the tech stack selected was a good choice even if there were some limitations.
\begin{itemize}
    \item WebGL was a good choice for a rendering API- It was highly performant and it's flexibility gave greater freedom to explore novel techniques and rendering designs. It's only downsides would be mixed support by Apple devices \cite[]{apple_metal} which limited the applications device accessibility.
    \item Using plain HTML for writing the view was not the best choice. Though in the context of this project it made sense to not bring in a UI framework, The student was already using many tools they haven't worked with before so it helped keep things simple- The agile approach also contributed to uncertainty on how useful a framework would be due to limited design done before development started. The Student would recommend future projects to consider options such as svelte or react to simplify UI development and simplify UI state bugs.
\end{itemize}

\subsection{Future Work} \label{futurework}
The application as is has potential for future development. Both in improving upon what the current application is already capable of doing and preparing it for further additions and expansions. The User testing 2 feature set and further improvements 2 feature set in particular (See \ref{UT2} and \ref{FI2} respectively) are a great place to start for debugging and polishing up the current functionality. Beyond that, there are also a number of 'Under the hood' issues that would be helpful to address for future development
\begin{itemize}
    \item Optimize rendering- There remain a lot of opportunities to optimize shaders and rendering in general. For instance, changing the default cube model (used to render data points) to something with less geometry could considerably speed up rendering. There is also opportunity to rework some aspects such as labels to focus more on saving memory vs cpu performance based on how Font + Model is made to interact. (See \ref{spr2}) There may also be opportunity to refactor the application to use WebGPU, which is a yet unreleased high-performance low-level graphics standard for the web inspired by Vulkan which should help increase performance even further. \cite{w3c_2023_webgpu} \cite{mozilla_webgpu}
    \item Refactor UI rendering- As seen during User Testing 2 (See \ref{usertestanalysis2}), the application UI has reached a level of complexity where the lack of a supporting UI library/Framework is starting to affect the quality of user experience as a consequence of more complex development. Any future UI work will benefit greatly from integrating a UI framework to handle components and state in particular. The Students recommendation would be React due to it's leading popularity among developers at the time being. \cite[]{stackoverflow_2021_stack}
    \item There is currently a limitation on what kind of controls can be done with the data. When data is uploaded, it is stored in memory as a variable. This means that the maximum amount of data directly depends on how much RAM the user's device has- and with the store being a simple variable, controls such as searching for a point cannot be optimized with database engine techniques such as indexing. This is probably at the moment the biggest hurdle holding back further big data support. A high-performance database would be ideal, either embedded into the application or as a server side component- with the server-side option providing better potential performance due to an independence from what the users hardware can handle.
\end{itemize}

Addressing these limitations would open up opportunities to effectively develop further additions for the project, some ideas by the student are as follows:

\begin{itemize}
    \item Address Velocity property of big data \cite{7918044} This will likely involve creating an API interface for the Application to connect to sensors or a rapidly changing database and optimizing the application to render as updates come in.
    \item Study further design techniques to reduce cognitive complexity of Multidimensional data. It's easy to add and render more dimensions but it's difficult to make it understandable enough to easily extract patterns. This point was looked at during the research phase, see \ref{academicbackground}.
    \item Add new chart types to visualize different data or the same data in another way, Radar charts were identified as a possible option during this project. (See \ref{otherresearch})
\end{itemize}

\subsection{Personal Development Appraisal}
The Student believes that this was a very productive project. learning wise, this project was the largest and longest piece of continuous they ever done and proved quite unique in the freedom it provided to plan and work as decided- which did in some cases prove challenging with the large amounts of decisions that came with it. The technologies used were all new for the student (Except for HTML, CSS and Python) and this project was the first time they used them. WebGL is particular was a huge departure from previous high Level API's used by the student and took considerable learning to grasp- Though taking the Graphics Module in Semester 1 which taught OpenGL greatly sped up the learning process.
With the completion of this project the student feels they have gained extensive practical experience in frontend, graphics development and more generally a better ability to  manage and plan their work.

\subsection{Summary}
This paper highlights a project undertaken by the student to develop a web-based, client-side scatterplot visualization application for Multidimensional data using graphics rendering technology (WebGL). The development process is shown from the beginning where the project background and initial concept (to create an accessible, Big data capable application) was formulated followed by formal specification and design decisions that drove an agile based, user testing driven development phase over the course of 3 months. A final review and analysis of the developed application follow with the created Applications limitations and suggestions for future development opportunities.
