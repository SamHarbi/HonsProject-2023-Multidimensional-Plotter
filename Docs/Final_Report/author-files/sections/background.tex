\section{Background}

The result of background research done for this project can be subdivided into two distinct sections being 'Related Work' and 'Tools and Technologies'. This was one of the earliest phases of the project whose results motivated the specific goals of this project- beyond just creating a Multidimensional plotting application as is required by the honours assessment that this project falls under, and explore what opportunities (in terms of tooling) exist to develop them.

\subsection{Related Work}
This section highlights the wider context that affected the structure and what the application was planned to do. The research identified here acted to directly influence what kind of application was developed, for what exact purpose and for what users.

\subsubsection{Project Context}
One of the first steps for this project was to place and understand the needs of this project through the lens of some wider context. By what metric should potential requirements be prioritised? and what those requirements should even be? Those are some of the questions whose answers will depend on the context that they are looked at. After some consideration, it was decided by the student to focus on the current business industry, and it’s needs as this context.

\begin{displayquote}
    This context (the business market) sets the aim of this project to then be an application that would be considered a valuable market entry compared to its competitors. It should also preferably, address some specific pain points of the market that are not fully covered by some other solution on the market.
\end{displayquote}

The alternative would also have been to look at this project from a more academic, literature driven view- where a project’s success would depend on more research oriented exploratory work. This alternative though, had some drawbacks that ultimately resulted in the selection of a Business Context. Mainly, a focus on existing market solutions allowed a more logical selection of features to develop for a practical development project. Although it should be stressed that this doesn’t mean that the academic background for this project was disregarded (See \ref{academicbackground})- It’s just that the focus was on the business-based requirements first and foremost.

\subsubsection{Market Research} \label{marketresearch}

With the context set- It was then considered that an analysis of the current options on the market would be a prudent next step to start understanding what a successful project should incorporate. This analysis was open-ended and focused on identifying key points that differentiated each product apart.

Detailed notes on each competitor can be found in the Appendix A1, which are although not exhaustive, nevertheless highlight some important aspect of the overall market- But the main important trends and highlights identified are the following:

Firstly, it was found that most solutions are highly technical and need at least some degree of programming knowledge to use. A rough relationship would be that the complexity of using the solution and its capability is inversely proportional. Feature rich solutions need programming experience while the ones that don’t, have more limited features, and are more likely to be not free to use. Obviously, there were outliers such as MATLAB which has both a UI interface and a programming interface, but this relationship still stands.

Another interesting point identified is that The Stack Overflow 2021 Developer Survey \cite[]{stackoverflow_2021_stack} shows that a large majority of the most used libraries and frameworks were for data analysis or data-based projects, specifically Python. Which is also the third most used language identified by the Survey \cite[]{stackoverflow_2021_stack}. Although these statistics don’t directly relate to visualization solutions, they help form an important background to the context in which visualization may be needed in. Which is that Python and its libraries are a highly prevalent option for undertaking data analysis projects, meaning that any visualization solution is likely to be part of a workflow that contains these tools.

But Solutions focusing on Visualization first and foremost seemed to be much fewer when compared to analysis solutions with visualization features. Those that were, were also much more likely to be highly technical. This point was found particularly important because data analysis and by extension visualization has been found to be needed in a wide field of industries, where programming expertise is not as widely common. And this need is expected to rise in the future with the advent of Industry 4.0- Which is a widely believed idea of increasing business productivity fueled by disruptive emergent technology in the near future \cite[]{GHOBAKHLOO2020119869}.

Taking all this into consideration, several key points were identified on how to structure a new development project to create a valuable solution from the market’s point of view:

\begin{itemize}
    \item Focus on accessibility first and foremost, including an easy-to-use interface that does not need programming experience.
    \item Focus on the visualization aspect, but either ensure that adequate analysis tools are available or importing or exporting data is easy and straight forward. Visualization is often only one step of a multi-step analysis project.
\end{itemize}

On the other hand, some thing’s to avoid are:

\begin{itemize}
    \item Focusing on the features but not accessibility. It is unreasonable to try to beat the current market players on features alone. You can’t get more feature-rich than a programming language such as Python, which is arguably one of the most used solutions on the market, See Appendix A1 for more information. It is important instead to create an alternative that is powerful enough but makes the process of data visualization much easier.
\end{itemize}

\subsubsection{Academic Background} \label{academicbackground}
A General study of research in the field was conducted with a focus on identifying any other similar projects such as this one undertaken by the student, and more importantly the wider context from an academic point of view. Which was followed by a comparison to previous market research in \ref{marketresearch}. This was done to give the student a better understanding of how visualization works in different settings and if there are any trends and patterns that can help prioritize functionality or uncover new opportunities.

This study was done by analyzing a small subset of important, state of the art research papers talking about data analytics and visualization. Some of these papers focus on market research which were particularly important to expand upon the market research done by the student themselves. The most important points identified are mentioned below (Consider this a continuation of the market research / section \ref{marketresearch} points above):

\begin{itemize}
    \item Current visualization techniques are not capable of handling big data \cite[]{7918044}. New solutions are being created and investigated by both industry and academia. Strategies for creating new visualization systems that handle each of the troubling aspects of big data have been put forward in \cite[]{9523950}. This is a big point that this project could contribute to fixing.
    \item No matter how good a visualization is- if it cannot be understood its value is nullified. There currently exists a gap between what computation is capable of creating and what can be easily grasped by a user. Naturally, there exists opportunity in creating solutions to close this gap by taking into consideration human cognitive psychology. \cite{olshannikova_2015_visualizing}
    \item Solving the problems to visualization poised by big data would require the combination of current visualization solutions merged with new technologies. \cite[]{olshannikova_2015_visualizing}
\end{itemize}

\subsubsection{Other Research}
Another important aspect of research that was done included a more technical look into general design points that could help drive the design of the application itself and how features are developed.

One specific aspect researched has been software planning methodologies. The student was aware of agile and waterfall techniques from their study but those techniques were usually applied in and made for multi-person teams, which naturally conflicted with the single-person structure of this project. Through this research, a technique called PSP \cite[]{493023} (Personal Software Process) was discovered. This is a methodology that allows a singular developer to apply a structured, continuously improving process to how they develop software. A further inspired process was also identified called PXP \cite[]{10.1145/1593105.1593127}, which was a fusion of Agile XP principles adapted to a single person team inspired by PSP.

Another notable finding in this phase, was an analysis of graphing and visualization techniques. In the \ref{marketresearch} Market Research phase a lot of different solutions with a large variety of graphing types were seen. An analysis of the most common ones seen, which also had some overlap with The seven basic tools of quality (A collection of seven charts that need minimal training to use \cite[]{ishikawa_1985_what} for analysis) was undertaken with the hopes of identifying the most suitable graphs in terms of flexibility at higher dimensions (3+). The full analysis document can be seen in Appendix A2 but in summary- It was found that Scatterplot's fit these needs the most while also being part of the Seven tools meaning increased accessibility in who could use the application. The other main options included the Radar chart which was suitable but not part of the Seven tools.

\subsection{Tools and Technologies} \label{toolsntech}
With the Application Context and Background set, the next step for this section is to highlight what technologies and design paradigms would be available for the development of this project followed by a comparison between them. No decisions were made during this stage of the project- The research and analysis recorded here act to justify and shortlist the final design in Section 4.

It should also be noted that obviously not all combinations of technologies mentioned below are viable. Usually deciding on one aspect would limit what can be selected for another. But nevertheless, each section below was looked at separately to fully understand the opportunity cost for each decision when going one route as opposed to another.

\subsubsection{Application Infastructure}
This section specifically focuses on the structural decisions in the design of an application that directly affect how it is created, run and in some cases what features are possible to implement. In general, the design options for this application in particular could be split into two groups.

\paragraph{Client-Side Run Application}
This is the simplest design. All code that makes up the application is run on the user’s device locally. No need for any server resources (other than for serving the initial code which even then may not need a server, Disc etc..) thus can be ran offline. But resources are limited to only what the user has and no way to synchronize data among multiple users- such as for user accounts, etc.

\paragraph{Server-Side or Full Stack Application}
This is a solution to some of the limitations mentioned for Client-side. Either have the client application be able to connect to a central (or server-less functions based) server and offload some tasks to it, or have the application be fully computed on-server with only static content responses being sent to the user. This design is naturally more complicated and often results in a split codebase among backend and frontend components. There is a need to provision computing resources, and more security considerations need to be made.

\subsubsection{Run Environment}
This can be defined as the environment where the application will run. The main options identified were the following:

\paragraph{.NET Environment}
.NET, which is an open-source developer platform for building Software mainly developed by Microsoft [4]. It is cross platform among Windows, MacOS and Linux (With support for iOS and Android using frameworks such .NET Multi-platform App UI / MAUI [5]). It provides a great rich set of libraries and tools for any kind of projects including both frontend and backend webapp needs and client ran applications (As long as .NET Framework is installed). .NET supports only 3 languages those being C\#, F\# and Visual Basic.

\paragraph{Node.js}
Node.js is aptly described on its website as an “an open-source, cross-platform JavaScript runtime environment.” [6] It is the most popular web technology among the Stack Overflow 2022 Survey respondents [7]. Although most commonly used for Server-side processing as a webserver, it can also be used for running local applications as long as it is installed on the client’s device (although it usually makes more sense to serve the app over a network). For package managers, npm is tightly linked with node.js and provides a huge repository of packages. With the addition of Web Assembly, which is a new standard assembly like language that acts as a compilation target from a wide range of other languages, it is technically possible to use almost any language with node.js that has a compiler to web assembly made.

\paragraph{Browser Engine}
Browser Engine’s- Modern browsers all have some JavaScript runtime engine, with V8 being the most widely used engine [8] Although all browsers are supposed to be cross-compatible, in practice this can vary, and some incompatibilities can arise. Browsers naturally only run client-side code but communicating with backend solutions is a common practice. Only JavaScript is natively supported and Web Assembly in “4 major browser engines”.

\paragraph{JRE (Java Runtime Environment)}
JRE, is a runtime environment that allows an application to be cross compatible between Operating systems by acting as a compatibility layer. It runs Java bytecode which can be compiled from a large variety of languages. A standard library is also available with the runtime environment. Commonly used for backend development but also capable of Frontend.

\paragraph{Compiled Program (OS only environment)}
Compiled Program, One of the most flexible options on the list. Languages such as C++ and C compile to machine code which are run directly on a user’s device. No inherit cross platform support and a need to recompile for each OS. Usually, higher performance due to lower abstraction. But a lower abstraction means more functionality needs to be managed by the programmer.

\subsubsection{Rendering Solution}
Being a project with visual requirements naturally meant that some rendering technology would be needed. Like with all other technologies mentioned thus far- there is a wide range of options that each have their own distinct design and ability. This section then aims to segment the available options by two factors at the minimum- One, the rendering solution must support 3D graphics in some capacity, and two, the solution must run and integrate with an application targeting one of the above-mentioned run environments. With those key requirements in mind, these were the identified options:

\paragraph{OpenGL}
OpenGL is a low-level cross-platform rendering API which can be traced back to a release date in 1992. Although no longer in active development as of 2017, OpenGL still remains highly supported across both newly releasing GPUs and older. This also includes mobile devices. In terms of language bindings, OpenGL is cross-language and can be called from almost any language and environment that has had bindings made for it.

\paragraph{Vulkan}
Vulkan is a cross-platform low level open standard rendering API that has superseded OpenGL in active development. It’s main driving improvement over OpenGL is lower overhead and more control over how code is run on the GPU. This greater control though does mean that development is more time consuming over OpenGL. Being much newer, with the initial release date being 2017, it is much less supported among older devices than OpenGL.

\paragraph{DirectX3D}
DirectX3D is a subset API of the proprietary DirectX family of multimedia APIs developed by Microsoft. It is a low-level API similar to OpenGL but created exclusively for Microsoft Windows, Xbox, and some Embedded Windows versions. In terms of languages, only C++ is supported targeting an executable on one of the above-mentioned systems.

\paragraph{WebGL}
A web integrated, JavaScript only low-level API version of OpenGL. It’s developed as an open web standard and is implemented in most widely used browsers, without the need to install it in any way for the client or developer. It’s a low-level API just like OpenGL and comes in two main versions based on OpenGL, 1.0 is based on the OpenGL ES 2.0 standard while 2.0 is based on OpenGL ES 3.0 which is less supported among older devices. As of writing, WebGL remains the lowest level of abstraction for running GPU code on the web.

\paragraph{Metal}
Metal is a low-level rendering API designed by Apple for their devices. It has a very limited set of supported hardware and is further limited to only iOS and MacOS for the Operating System. It was created by apple to replace OpenGL on their hardware which was depreciated in 2018. In terms of language bindings, Objective-C and C++ are the only supported options.

\subsubsection{Optional Abstractions for Rendering Solutions}
The underlying rendering technologies mentioned in the last section, although usable on their own, can be offloaded to be managed by a higher-level framework or engine. Although it must be stressed that the inherit properties of the underlying rendering solution still has an effect on the application being developed.
The main reason to bring in an abstraction would usually be to lower development complexity, and in turn time. But the drawback often results in less flexibility regarding what can be made and lower performance.

\paragraph{Three.js}
Three.js is a JavaScript library that abstracts WebGL to simplify the creation of 3D graphics. It is highly extensive and offers a wide array of useful functions and constructs to that end. It is available as a npm package and can be easily integrated into a web-based project. If comparing to WebGL though, which is already a part of the browser, three.js contributes to a much bigger application download footprint. It also suffers from the same general relationship mentioned at the start of this section, development is greatly simplified but a lot of the flexibility and performance is lost.

\paragraph{Unity}
Unity, although first and foremost a game engine, can be applied to any development project where high performance graphics are needed. Unity supports DirectX, OpenGL, Vulkan and Metal- with only a few settings changes. A wide range of target devices can also then be built for. It is also possible to target WebGL as a platform in a similar fashion for web applications. Beyond this great cross-device and API support- Unity has an extensive collection of tools and APIs to streamline development. Although it should be mentioned that web applications are not as well supported, especially on mobile devices. The engine also only supports C\# as the only option.

\subsubsection{Languages}
There are a large variety of potential languages that could be used for writing this program. The following were selected based on how extensively the languages are supported in previously mentioned Rendering solutions and Run Environment Sections. And some key exceptions that were considered to be important for consideration by the student.

\paragraph{C++}
A well know general-purpose compiled programming language most often used in performance constraint applications. It is Object-Oriented but can be used to also write functional code. It is arguably the de-facto standard for systems programming.

\paragraph{Rust}
Rust is a relatively new general-purpose compiled programming language often used in systems programming. It is a modern alternative to C++ that combines its high performance with a multitude of modern features such as a default package manager, and numerous safety features especially for memory (While still keeping a low performance impact).

\paragraph{Java}
Java is a general-purpose, object-oriented focused programming language that compiles to Java Bytecode that can be run on any device running a JVM (Java Virtual Machine).

\paragraph{C\#}
C\# is a general-purpose, multi-paradigm language developed by Microsoft. Although it is technically possible to compile to machine code, it is most often compiled to run on a .NET environment.

\paragraph{JavaScript}
JavaScript is the natively supported programming language on the web. It is multi-paradigm and is just-in-time compiled. It requires a runtime which is often a browser or dedicated runtime environments such as Node.js

\paragraph{TypeScript}
Typescript is a superset version of JavaScript developed by Microsoft that a number of features such as stricter syntax with the ability to have types. It transpiles to JavaScript which is then run as described above in 2.2.5.5.

\subsection{Summary}

The background research mentioned in this section acted to ground the other-wise open-ended project to a set of goals, both mandatory and optional, to create an impactful and original project. The following paragraph then summarizes the goals previously brought up into a short, guiding statement.

\begin{displayquote}
    The purpose of this project is to create a new market entry application for the creation of visualizations. This application would focus on user accessibility first and foremost, to be an application that is usable without programming knowledge. It will allow Multidimensional data to be plotted through a Scatterplot plot and possibly more chart options. In it's creation it will not ignore the current problems poised by Big Data and will attempt to support it thorough the application of state of the art techniques to support it's visualisations or be designed in such a way as to support their future integration into the application.
\end{displayquote}


Otherwise the research mentioned thus far also sets an origin from which all further decisions are justified by. For the final selection and analysis of the tech stack chosen by the student to develop the application, see section \ref{techstack}. To see how market research affected the application design, see section \ref{uxdesign}. These are just two of the bigger sections building upon the research set here- thus, other references linking here are to be expected as different parts of the project are focused on.


