% Implementation Tables 



\begin{table*}[h]
    \begin{tabularx}{\textwidth}{ | X | X | X | }
        \hline
        ID & Title                                                                                               & Estimate (Relative) \\
        \hline
        3  & As a User, I want to be able to view a scatterplot of data set against an axis with accurate scales & 5                   \\
        \hline
        11 & As a User, I want all interactive parts of the Application to be visible at all times               & 1                   \\
        \hline
    \end{tabularx}
    \caption{MVP Feature Set, Sprint 1}
    \label{sprint1}
\end{table*}

\begin{table*}[h]
    \begin{tabularx}{\textwidth}{ | X | X | X | }
        \hline
        ID & Title                                                                                                          & Estimate (Relative) \\
        \hline
        8  & As a User, I want the application to have easy to use on screen controls for interacting with the application  & 3                   \\
        \hline
        1  & As a User, I want to be able to import a dataset into the application to graph it in a 3 dimension scatterplot & 3                   \\
        \hline
        18 & The Axis should scale with the values shown                                                                    & 5                   \\
        \hline
        7  & As a User, I want to be able to rotate the 3D Scatterplot around all 3D axis individually                      & 5                   \\
        \hline
    \end{tabularx}
    \caption{MVP Feature Set, Sprint 2}
    \label{sprint2}
\end{table*}

\begin{table*}[h]
    \begin{tabularx}{\textwidth}{ | X | X | X | }
        \hline
        ID & Title                                                                                                               & Estimate (Relative) \\
        \hline
        2  & As a User, I want to be able to navigate around the generated graph in 3D space while having the axis stay accurate & 6                   \\
        \hline
        5  & As a User, I want to move the 3D view of the scatterplot in 3D using on screen controls                             & 3                   \\
        \hline
        6  & As a User, I want to be able to zoom in and out of the 3D Scatterplot                                               & 10                  \\
        \hline
        9  & As a User, I want to be able to access the application on landscape screens of different sizes                      & 2                   \\
        \hline
        24 & UI doesn't scale very well and not all controls visible                                                             & 3                   \\
        \hline
    \end{tabularx}
    \caption{MVP Feature Set, Sprint 3}
    \label{sprint3}
\end{table*}

\begin{table*}[h]
    \begin{tabularx}{\textwidth}{ | X | X | X | }
        \hline
        ID & Title                                                                                                  & Estimate (Relative) \\
        \hline
        10 & As a User, I want to be able to view instructions within the application on how to use the application & 2                   \\
        \hline
        26 & The User should be able to change the size of the data points                                          & 2                   \\
        \hline
        29 & The User should have the ability to use Mouse Controls for transformation controls                     & 3                   \\
        \hline
        30 & There should be a visible way to identify negative axis values                                         & 3                   \\
        \hline
    \end{tabularx}
    \caption{MVP Feature Set, Sprint 4}
    \label{sprint4}
\end{table*}

\begin{table*}[hbt!]
    \begin{tabularx}{\textwidth}{ | X | X | X | }
        \hline
        ID & Title                                                                               & Estimate (Relative) \\
        \hline
        43 & Refactor Needed (MVC Pattern potential)                                             & 5                   \\
        \hline
        41 & Each Axis should be named / display a name based on the column names in the dataset & 4                   \\
        \hline
    \end{tabularx}
    \caption{Backlog Tasks, Sprint 5}
    \label{sprint5}
\end{table*}

\begin{table*}[hbt!]
    \begin{tabularx}{\textwidth}{ | X | X | X | X | X | }
        \hline
        ID & Title                                                                                                                                         & Estimate (Relative) & MoSCoW & Feature Set          \\
        \hline
        42 & Create a more robust zoom not limited to only 2 levels                                                                                        & 7                   & Must   & Further improvements \\
        \hline
        52 & As a Tester, I want data labels to be always visible even when zooming                                                                        & 2                   & Must   & User Testing 1       \\
        \hline
        54 & Mentioned by Testers, If the slice viewed is not at origin, then using the data zoom should not result in data points moving out of the graph & 5                   & Must   & User Testing 1       \\
        \hline
        53 & As a Tester, I want to be able to click on a data point and have the value of that point be displayed                                         & 8                   & Must   & User Testing 1       \\
        \hline
        57 & As a Tester, Graph fully doesn't render in rare cases                                                                                         & 1                   & Must   & User Testing 1       \\
        \hline
        70 & Have a Favicon                                                                                                                                & 1                   & Must   & Further improvements \\
        \hline
    \end{tabularx}
    \caption{Backlog Tasks, Sprint 6}
    \label{sprint6}
\end{table*}

\begin{table*}[hbt!]
    \begin{tabularx}{\textwidth}{ | X | X | X | X | X | }
        \hline
        ID & Title                                                                                         & Estimate (Relative) & MoSCoW & Feature Set          \\
        \hline
        58 & As a Tester, I want to be able to return the graph to it's starting point with a button click & 1                   & Must   & User Testing 1       \\
        \hline
        63 & The application should be able to use color as an extra dimension                             & 6                   & Must   & Further improvements \\
        \hline
        64 & The application should have an option to apply columns to different dimensions                & 9                   & Should & Further improvements \\
        \hline
        74 & The application should have a view where the data table can be viewed                         & 5                   & Should & Further improvements \\
        \hline
    \end{tabularx}
    \caption{Backlog Tasks, Sprint 7}
    \label{sprint7}
\end{table*}

\begin{table*}[hbt!]
    \begin{tabularx}{\textwidth}{ | X | X | X | X | X | }
        \hline
        ID & Title                                                                                                                    & Estimate (Relative) & MoSCoW & Feature Set    \\
        \hline
        47 & As a Tester, I want to be able to zoom in and out of the graph using the scroll wheel                                    & 1                   & Should & User Testing 1 \\
        \hline
        48 & As a Tester, I want to be able to focus on a single axis through the click of a button- i.e View the graph from one side & 2                   & Should & User Testing 1 \\
        \hline
    \end{tabularx}
    \caption{Backlog Tasks, Sprint 8}
    \label{sprint8}
\end{table*}
