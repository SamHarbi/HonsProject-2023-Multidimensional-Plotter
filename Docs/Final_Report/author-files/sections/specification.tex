\section{Specification}

\subsection{Development Methodology}
PXP as introduced in the research phase was selected for its well defined agile based methodology that was well suited to a single-person development team. Agile in general though, was selected to allow user feedback to drive design and development which was often collected. The details of the methodology are mainly based on this paper[]
As is a fundamental principle of agile, this methodology was adapted to the project at hand with some main changes being:
\begin{itemize}
    \item No automated testing setup at the start of the project. Graphics development is fairly tricky to automate the testing of as the large majority of testing is completely visual. It is still possible and was considered in section X but was chosen to set aside to rather focus on development and proving that the technology stack was suitable before any more work was committed on it.
    \item Allow refactoring to be raised at any point, also allow grouping of similar user stories to be refactored together.
    \item MoSCoW and Cost factors agile ceremonies were considered for the project’s user stories.
\end{itemize}

Otherwise, the development process was followed as written. To have a short summary, development consisted of identifying and analyzing user stories, separating them into feature sets, planning iterations,

\subsection{Requirements Gathering}
With the goals that the developed application needs to achieve set as per the research phase- The next step was to create a concrete plan on what exact requirements / features would contribute towards achieving those goals. To do that, two main ideas were identified by the student.

The first idea was to conduct Interviews with individuals who commonly use visualization tools or do general data science work. This would provide key insight into what real users of visualization technology think is key for a successful application to have. These requirements would constitute the first phase of the project and set a strong start to either a waterfall type methodology or an agile project.
But there were a couple of downsides that cancelled out this idea at that time:
\begin{itemize}
    \item With no initial product to focus insight into actionable requirements- the interview process is more likely to return conflicting or infeasible requirements.
    \item It might be difficult to offer insight that is not generic for the same reasons. Application should plot data vs Application should do it like this instead of like this. With the latter insight being much more valued.
    \item Time and access to experts is very valuable and needs extensive preparation. It is important to make the most of it, which the student didn’t feel like they could do at that time.
\end{itemize}

The next idea to identify requirements was a two-step process, and one which was inspired by the development methodology chosen to be followed by the student in section X. The student took on the role of the client to create a client brief using the insight gained during the research phase. This brief can be read in full in Appendix A but in short, creates a written source document describing the minimum simplest application that would need to be created. This document then served as the basis from which user stores were extracted.