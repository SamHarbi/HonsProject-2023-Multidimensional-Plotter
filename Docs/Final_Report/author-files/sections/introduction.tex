\section{Introduction}


Data visualization is an important step in the data analysis process. Whether used on it's own as a tool for analysis, or as a final stop to clearly articulate and display results. It is in spirit a tool for communication, allowing abstract and complex ideas to be easily communicated to a wide audience, usually requiring minimal expertise to understand (Based on the visualization design done). It's ubiquity across all industries is then no surprise \cite{olshannikova_2015_visualizing}. Yet limitations and shortcomings exist, the accessibility of tools to create visualizations are often complex and at odds with the accessibility of the created artifacts. This limits creators by requiring them to have considerable knowledge of programming fundamentals, or be limited by what they are able to create. This is further compounded with 3D+ Multidimensional data visualization, Which has an even smaller selection of accessible tools.

Another challenge, when considering all tools regardless of user accessibility, has been the effect of Big Data on the user requirements of these applications. The ever increasing size and complexity of datasets doesn't invalidate the value provided by visualization to smaller datasets. In fact, It's importance as a communication aid is greater than ever before. Yet, current visualization tools and techniques cannot keep up with this requirement \cite{7918044}. In a way, this is just another accessibility problem- an inaccessibility of certain data types.

The aim of this project then was try to address the roots of these inaccessibility problems by creating a new market contender application focused with this problem in mind. This consisted of an academic long software development project undertaken by the student- with a resultant Multidimensional Scatterplot application created. This paper aims to highlight the decisions made in design and development while also recording the entire process followed by a look back on if the project was successful.









