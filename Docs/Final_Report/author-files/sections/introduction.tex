\section{Introduction}


Data visualisation is an important step in the data analysis process. An importance which is only further compounded by the rise of Big Data and the subsequent need for extracting insight from exponentially larger and more complex data sets. This insight, whether it’s used to highlight results in an easy to digest manner or to identify patterns otherwise hidden by the complexity of a dataset- can bring value across a project’s team, with each member being able to extract value specific to their own tasks.

For all members working on a project (With the word project being used in the context of any work done that uses data within an industrial or academic sector) visualisation is the main “window” through which insight and thus value is accessed. Analysis is done towards some end- and that end must result in the return of something- which often must be visually consumed (See Big Data above). In many cases visualisation in itself can act as a form of analysis, both creating insight and communicating it when applied to a dataset.

To be more specific, it has been found that there is a definite market among both technical and non-technical professionals for tools that can unlock the results of analysis.

The aim of this project was to then create an accessible, web-based application that can be used by anyone to quickly plot data in multiple dimensions, regardless of programming skill. This paper highlights the decisions that lead to this final
